\documentclass[9pt,a4paper,]{extarticle}

\usepackage{f1000_styles}

\usepackage[pdfborder={0 0 0}]{hyperref}

\usepackage[numbers]{natbib}
\bibliographystyle{unsrtnat}


%% maxwidth is the original width if it is less than linewidth
%% otherwise use linewidth (to make sure the graphics do not exceed the margin)
\makeatletter
\def\maxwidth{ %
  \ifdim\Gin@nat@width>\linewidth
    \linewidth
  \else
    \Gin@nat@width
  \fi
}
\makeatother


% disable code chunks background
%\renewenvironment{Shaded}{}{}

% disable section numbers
\setcounter{secnumdepth}{0}

%% added by MLS, this is not in the F1000 style by default %%

\hypersetup{unicode=true,
            pdftitle={BASiCS workflow: a step-by-step analysis of expression variability using single cell RNA sequencing data},
            pdfkeywords={Single-cell RNA sequencing, expression variability, transcriptional noise, differential expression testing},
            colorlinks=true,
            linkcolor=Maroon,
            citecolor=Blue,
            urlcolor=Orange,
            breaklinks=true}

%% End added by MLS %%

\setlength{\parindent}{0pt}
\setlength{\parskip}{6pt plus 2pt minus 1pt}



\begin{document}
\pagestyle{front}

\title{BASiCS workflow: a step-by-step analysis of expression variability using single cell RNA sequencing data}

\author[1,2]{Nils Eling\thanks{\ttfamily eling@ebi.ac.uk}}
\author[3]{Alan O'Callaghan}
\author[1,2]{John C. Marioni}
\author[3,4]{Catalina A. Vallejos\thanks{\ttfamily catalina.vallejos@igmm.ed.ac.uk}}
\affil[1]{European Molecular Biology Laboratory, European Bioinformatics Institute, Wellcome Trust Genome Campus, Hinxton, Cambridge CB10 1SD, UK}
\affil[2]{Cancer Research UK Cambridge Institute, University of Cambridge, Li Ka Shing Centre, Cambridge, CB2 0RE, UK}
\affil[3]{MRC Human Genetics Unit, Institute of Genetics \& Molecular Medicine, University of Edinburgh, Western General Hospital, Crewe Road, Edinburgh, EH4 2XU, UK}
\affil[4]{The Alan Turing Institute, British Library, 96 Euston Road, London, NW1 2DB, UK}

\maketitle
\thispagestyle{front}

\begin{abstract}
Cell-to-cell gene expression variability is an inherent feature of complex
biological systems, such as immunity and development. Single-cell RNA
sequencing is a powerful tool to quantify this heterogeneity, but it is prone
to strong technical noise. In this article, we describe a step-by-step
computational workflow which uses the BASiCS Bioconductor package to robustly
quantify expression variability within and between known groups of cells (such
as experimental conditions or cell types). BASiCS uses an integrated framework
for data normalisation, technical noise quantification and downstream
analyses, whilst propagating statistical uncertainty across these steps.
Within a single seemingly homogeneous cell population, BASiCS can identify
highly variable genes that exhibit strong heterogeneity as well as lowly
variable genes with stable expression. BASiCS also uses a probabilistic
decision rule to identify changes in expression variability between cell
populations, whilst avoiding confounding effects related to differences in
technical noise or in overall abundance. Using two publicly available
datasets, we guide users through a complete pipeline which includes
preliminary steps for quality control as well as data exploration
using the scater and scran Bioconductor packages. Data for the first case
study was generated using the Fluidigm@ C1 system, in which extrinsic
spike-in RNA molecules were added as a control. The second dataset was
generated using a droplet-based system, for which spike-in RNA is not
available. This analysis provides an example, in which differential
variability testing reveals insights regarding a possible early cell fate
commitment process. The workflow is accompanied by a Docker image that
ensures the reproducibility of our results.
\end{abstract}

\section*{Keywords}
Single-cell RNA sequencing, expression variability, transcriptional noise, differential expression testing


\clearpage
\pagestyle{main}

\hypertarget{quantifying-cell-to-cell-transcriptional-variability---basics}{%
\subsection{\texorpdfstring{Quantifying cell-to-cell transcriptional variability - \texttt{BASiCS}}{Quantifying cell-to-cell transcriptional variability - BASiCS}}\label{quantifying-cell-to-cell-transcriptional-variability---basics}}

The \emph{\href{https://bioconductor.org/packages/3.11/BASiCS}{BASiCS}} package implements a Bayesian hierarchical framework
which borrows information across all genes and cells to robustly quantify
transcriptional variability \citep{Vallejos2015BASiCS}.
Similar to the approach adopted in \emph{\href{https://bioconductor.org/packages/3.11/scran}{scran}}, \emph{\href{https://bioconductor.org/packages/3.11/BASiCS}{BASiCS}}
infers cell-specific global scaling normalisation parameters.
However, instead of inferring these as a pre-processing step,
\emph{\href{https://bioconductor.org/packages/3.11/BASiCS}{BASiCS}} uses an integrated approach in which data normalisation
and downstream analyses are performed simultaneously --- whilst propagating
statistical uncertainty.
To quantify technical noise, the original implementation of
\emph{\href{https://bioconductor.org/packages/3.11/BASiCS}{BASiCS}} uses information from extrinsic spike-in molecules as
control features, but the model has been extended to address situations in which
spike-ins are not available \citep{Eling2018}.

\emph{\href{https://bioconductor.org/packages/3.11/BASiCS}{BASiCS}} summarises the distribution of gene expression through
gene-specific \emph{mean} (\(\mu_i\)) and \emph{over-dispersion} (\(\delta_i\)) parameters.
Mean parameters \(\mu_i\) quantify the overall expression for each gene \(i\)
across the population of cells under study.
Instead, \(\delta_i\) captures the excess of variability that is observed with
respect to what would be expected in a homogeneous cell population, after taking
into account technical noise.
These are used as a proxy to quantify transcriptional variability.
Moreover, to account for the strong association that is typically observed
between mean expression and over-dispersion estimates, we recently introduced
gene-specific \emph{residual over-dispersion} parameters \(\epsilon_i\) \citep{Eling2018}.
Similar to DM values implemented in \emph{\href{https://bioconductor.org/packages/3.11/scran}{scran}}, these are defined as
deviations with respect to an overall regression trend that captures the
relationship between mean and over-dispersion values.

Parameter estimation is performed using an adaptive Metropolis within Gibbs
algorithm \citep{Roberts2009}.
This is implemented in the \texttt{BASiCS\_MCMC} function, which can be run using four
different major settings (see Table 1).
The default {[}describe the overall setting and recommend regression = true{]}.

If spike-in counts are availabe and should be used to estimate technical noise,
the parameter \texttt{WithSpikes} is set to \texttt{TRUE} (default).
If the regression between over-dispersion and mean expression should be
performed, the \texttt{Regression} parameters is set to \texttt{TRUE} (default).
If the user decides to set \texttt{Regression\ =\ FALSE}, \texttt{BASiCS} will not estimate
the regression trend, and will not supply the residual over-dispersion
parameters \(\epsilon_i\).

\begin{table}[htbp]
\caption{Four settings available for the the \texttt{BASiCS\_MCMC} function.}
\centering
\begin{tabledata}{@{}lll@{}}
\header & No regression & Regression\\
\row Using spike-in reads & \texttt{WithSpikes\ =\ TRUE} & \texttt{WithSpikes\ =\ TRUE}\\
\row & \texttt{Regression\ =\ FALSE} & \texttt{Regression\ =\ TRUE}\\
\row No spike-ins available & \texttt{WithSpikes\ =\ FALSE} & \texttt{WithSpikes\ =\ FALSE}\\
\row & \texttt{Regression\ =\ FALSE} & \texttt{Regression\ =\ TRUE}\\
\end{tabledata}
\end{table}

The \texttt{BASiCS\_MCMC} function returns a \texttt{BASiCS\_Chain} object, which can be used
for further downstream analyses, many of which are detailed in this workflow.
These objects contain draws from Markov chain Monte Carlo (MCMC) samplers,
which are used to infer the posterior distribution over the model parameters
\citep{Smith1993}.
Briefly, the posterior distribution quantifies how probable different parameter
values are given the observed data. However, before assessing the posterior
distribution, we must first ensure that the MCMC sampler has converged to
its stationary distribution, and has sampled efficiently from this distribution
\citep{Cowles1996}. If these conditions are not met, then the estimated parameters
may be inaccurate. The \emph{\href{https://CRAN.R-project.org/package=coda}{coda}} CRAN package contains a variety of
functions to assess the convergence of a sampled MCMC chain.
To use \texttt{coda} functions, the individual chains returned by \texttt{BASiCS} need to be
transformed into a MCMC object that \texttt{coda} recognises using the \texttt{coda::mcmc}
function. \texttt{BASiCS} also offers a number of functions to visualise and assess the
convergence of MCMC chains. In particular, we will use
\texttt{BASiCS\_EffectiveSize} and \texttt{BASiCS\_DiagPlot} to calculate and visualise the
effective sample size generated by the MCMC samplers.

contains an assembly of
functions to estimate and analyse gene- and cell-specific model parameters
\citep{Vallejos2015BASiCS, Vallejos2016, Eling2018}.

{[}talk about mean and over-dispersion parameters; how can these be used to select hvg/lvg;
then mention the ability to perform differential testing;
finally the extention to account for mean/over-dispersion{]}

\hypertarget{other-steps-title-tbc}{%
\subsection{Other steps {[}title tbc{]}}\label{other-steps-title-tbc}}

The \emph{\href{https://bioconductor.org/packages/3.11/goseq}{goseq}} Bioconductor package offers functions to detect the
enrichment of gene ontologies (GOs) among user-specified gene sets \citep{Young2010}.
Furthermore, \texttt{goseq} corrects for gene length biases, which is useful for full
length scRNA-seq data as highlighted in the first section.
In this workflow, we will use \texttt{goseq} to detect GO enrichment among
differentially expressed sets of genes.

For downstream analysis, such as GO enrichment analysis or the biological
interpretation of individual genes, we need to (i) link each gene's ID to its
symbol and (ii) calculate each gene's length.
For the first task, the \emph{\href{https://bioconductor.org/packages/3.11/biomaRt}{biomaRt}} Bioconductor package annotates a
wide range of gene and gene product identifiers \citep{Durinck2005} by accessing the
BioMart software suite (\url{http://www.biomart.org}).
We can use \texttt{biomaRt} to link the \textbf{Mus musculus} gene IDs and to their gene
symbols (also referred to as `gene name'):

{\small\bibliography{Workflow.bib}}

\end{document}
