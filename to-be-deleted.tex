\documentclass[9pt,a4paper,]{extarticle}

\usepackage{f1000_styles}

\usepackage[pdfborder={0 0 0}]{hyperref}

\usepackage[numbers]{natbib}
\bibliographystyle{unsrtnat}


%% maxwidth is the original width if it is less than linewidth
%% otherwise use linewidth (to make sure the graphics do not exceed the margin)
\makeatletter
\def\maxwidth{ %
  \ifdim\Gin@nat@width>\linewidth
    \linewidth
  \else
    \Gin@nat@width
  \fi
}
\makeatother

\usepackage{color}
\usepackage{fancyvrb}
\newcommand{\VerbBar}{|}
\newcommand{\VERB}{\Verb[commandchars=\\\{\}]}
\DefineVerbatimEnvironment{Highlighting}{Verbatim}{commandchars=\\\{\}}
% Add ',fontsize=\small' for more characters per line
\usepackage{framed}
\definecolor{shadecolor}{RGB}{248,248,248}
\newenvironment{Shaded}{\begin{snugshade}}{\end{snugshade}}
\newcommand{\AlertTok}[1]{\textcolor[rgb]{0.94,0.16,0.16}{#1}}
\newcommand{\AnnotationTok}[1]{\textcolor[rgb]{0.56,0.35,0.01}{\textbf{\textit{#1}}}}
\newcommand{\AttributeTok}[1]{\textcolor[rgb]{0.77,0.63,0.00}{#1}}
\newcommand{\BaseNTok}[1]{\textcolor[rgb]{0.00,0.00,0.81}{#1}}
\newcommand{\BuiltInTok}[1]{#1}
\newcommand{\CharTok}[1]{\textcolor[rgb]{0.31,0.60,0.02}{#1}}
\newcommand{\CommentTok}[1]{\textcolor[rgb]{0.56,0.35,0.01}{\textit{#1}}}
\newcommand{\CommentVarTok}[1]{\textcolor[rgb]{0.56,0.35,0.01}{\textbf{\textit{#1}}}}
\newcommand{\ConstantTok}[1]{\textcolor[rgb]{0.00,0.00,0.00}{#1}}
\newcommand{\ControlFlowTok}[1]{\textcolor[rgb]{0.13,0.29,0.53}{\textbf{#1}}}
\newcommand{\DataTypeTok}[1]{\textcolor[rgb]{0.13,0.29,0.53}{#1}}
\newcommand{\DecValTok}[1]{\textcolor[rgb]{0.00,0.00,0.81}{#1}}
\newcommand{\DocumentationTok}[1]{\textcolor[rgb]{0.56,0.35,0.01}{\textbf{\textit{#1}}}}
\newcommand{\ErrorTok}[1]{\textcolor[rgb]{0.64,0.00,0.00}{\textbf{#1}}}
\newcommand{\ExtensionTok}[1]{#1}
\newcommand{\FloatTok}[1]{\textcolor[rgb]{0.00,0.00,0.81}{#1}}
\newcommand{\FunctionTok}[1]{\textcolor[rgb]{0.00,0.00,0.00}{#1}}
\newcommand{\ImportTok}[1]{#1}
\newcommand{\InformationTok}[1]{\textcolor[rgb]{0.56,0.35,0.01}{\textbf{\textit{#1}}}}
\newcommand{\KeywordTok}[1]{\textcolor[rgb]{0.13,0.29,0.53}{\textbf{#1}}}
\newcommand{\NormalTok}[1]{#1}
\newcommand{\OperatorTok}[1]{\textcolor[rgb]{0.81,0.36,0.00}{\textbf{#1}}}
\newcommand{\OtherTok}[1]{\textcolor[rgb]{0.56,0.35,0.01}{#1}}
\newcommand{\PreprocessorTok}[1]{\textcolor[rgb]{0.56,0.35,0.01}{\textit{#1}}}
\newcommand{\RegionMarkerTok}[1]{#1}
\newcommand{\SpecialCharTok}[1]{\textcolor[rgb]{0.00,0.00,0.00}{#1}}
\newcommand{\SpecialStringTok}[1]{\textcolor[rgb]{0.31,0.60,0.02}{#1}}
\newcommand{\StringTok}[1]{\textcolor[rgb]{0.31,0.60,0.02}{#1}}
\newcommand{\VariableTok}[1]{\textcolor[rgb]{0.00,0.00,0.00}{#1}}
\newcommand{\VerbatimStringTok}[1]{\textcolor[rgb]{0.31,0.60,0.02}{#1}}
\newcommand{\WarningTok}[1]{\textcolor[rgb]{0.56,0.35,0.01}{\textbf{\textit{#1}}}}

% disable code chunks background
%\renewenvironment{Shaded}{}{}

% disable section numbers
\setcounter{secnumdepth}{0}

%% added by MLS, this is not in the F1000 style by default %%

\hypersetup{unicode=true,
            pdftitle={BASiCS workflow: a step-by-step analysis of expression variability using single cell RNA sequencing data},
            pdfkeywords={Single-cell RNA sequencing, expression variability, transcriptional noise, differential expression testing},
            colorlinks=true,
            linkcolor=Maroon,
            citecolor=Blue,
            urlcolor=Orange,
            breaklinks=true}

%% End added by MLS %%

\setlength{\parindent}{0pt}
\setlength{\parskip}{6pt plus 2pt minus 1pt}



\begin{document}
\pagestyle{front}

\title{BASiCS workflow: a step-by-step analysis of expression variability using single cell RNA sequencing data}

\author[1,2]{Nils Eling\thanks{\ttfamily eling@ebi.ac.uk}}
\author[3]{Alan O'Callaghan}
\author[1,2]{John C. Marioni}
\author[3,4]{Catalina A. Vallejos\thanks{\ttfamily catalina.vallejos@igmm.ed.ac.uk}}
\affil[1]{European Molecular Biology Laboratory, European Bioinformatics Institute, Wellcome Trust Genome Campus, Hinxton, Cambridge CB10 1SD, UK}
\affil[2]{Cancer Research UK Cambridge Institute, University of Cambridge, Li Ka Shing Centre, Cambridge, CB2 0RE, UK}
\affil[3]{MRC Human Genetics Unit, Institute of Genetics \& Molecular Medicine, University of Edinburgh, Western General Hospital, Crewe Road, Edinburgh, EH4 2XU, UK}
\affil[4]{The Alan Turing Institute, British Library, 96 Euston Road, London, NW1 2DB, UK}

\maketitle
\thispagestyle{front}

\begin{abstract}
Cell-to-cell gene expression variability is an inherent feature of complex
biological systems, such as immunity and development. Single-cell RNA
sequencing is a powerful tool to quantify this heterogeneity, but it is prone
to strong technical noise. In this article, we describe a step-by-step
computational workflow which uses the BASiCS Bioconductor package to robustly
quantify expression variability within and between known groups of cells (such
as experimental conditions or cell types). BASiCS uses an integrated framework
for data normalisation, technical noise quantification and downstream
analyses, whilst propagating statistical uncertainty across these steps.
Within a single seemingly homogeneous cell population, BASiCS can identify
highly variable genes that exhibit strong heterogeneity as well as lowly
variable genes with stable expression. BASiCS also uses a probabilistic
decision rule to identify changes in expression variability between cell
populations, whilst avoiding confounding effects related to differences in
technical noise or in overall abundance. Using two publicly available
datasets, we guide users through a complete pipeline which includes
preliminary steps for quality control as well as data exploration
using the scater and scran Bioconductor packages. Data for the first case
study was generated using the Fluidigm@ C1 system, in which extrinsic
spike-in RNA molecules were added as a control. The second dataset was
generated using a droplet-based system, for which spike-in RNA is not
available. This analysis provides an example, in which differential
variability testing reveals insights regarding a possible early cell fate
commitment process. The workflow is accompanied by a Docker image that
ensures the reproducibility of our results.
\end{abstract}

\section*{Keywords}
Single-cell RNA sequencing, expression variability, transcriptional noise, differential expression testing


\clearpage
\pagestyle{main}

\begin{Shaded}
\begin{Highlighting}[]
\KeywordTok{library}\NormalTok{(SingleCellExperiment)}
\KeywordTok{library}\NormalTok{(scater)}
\KeywordTok{library}\NormalTok{(scran)}
\KeywordTok{library}\NormalTok{(BASiCS)}
\end{Highlighting}
\end{Shaded}

\hypertarget{qc-and-exploratory-analysis}{%
\subsection{QC and exploratory analysis}\label{qc-and-exploratory-analysis}}

These QC metrics can also be visualised with respect to cell-level metadata.
For example, the following figures show how these metrics relate to the
experimental conditions (active vs unstimulated) and to the different mice
from which cells were collected.

For this purpose, we apply a Principal Component Analysis (PCA) after \emph{\href{https://bioconductor.org/packages/3.11/scran}{scran}} normalisation.

In the next step, we will use the and
\emph{\href{https://bioconductor.org/packages/3.11/scater}{scater}} Bioconductor packages for initial normalisation and
visualisation of quality metrics.
The normalisation at this point is needed to avoid biases in the visualisation
due to differences in the mRNA content between cells.

{\small\bibliography{Workflow.bib}}

\end{document}
